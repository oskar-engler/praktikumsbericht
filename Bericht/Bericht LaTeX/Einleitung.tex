\chapter*{Einleitung}\thispagestyle{fancy}\markboth{Einleitung}{1}
\addcontentsline{toc}{chapter}{Einleitung}


\large {Die Firma Belectric GmbH wurde im 2001 gegr�ndet und hat seinen Standort im Kolitzheim. Seit dem wurden �ber 1,5 GWp Solarleistung weltweit installiert. Die Belectric GmbH wurde damit zu einem Weltmarktf�hrer im Bereich der Installation von Freifl�chensolarkraftwerken. Es werden neue und innovative Technologien bei der Installation umgesetzt. Weltweit hat die Belectric �ber 1600 Besch�ftigte, die in den Bereichen von Wartung und Anlagenbau bis hin zu Forschung und Entwicklung t�tig sind. \\

\begin{figure}[htbp]
  \centering
     \includegraphics[width=0.7\textwidth]{images/Belectric-Logo.png}
  \caption{Logo Belectric}
  \label{fig:Logo Belectric}
\end{figure}

Die Firma Adensis GmbH, mit dem Standort im Dresden, geh�rt zu den Entwicklungs- und Forschungsgruppen der Belectric GmbH. Sie wurde 2006 gegr�ndet und ist seit dem ein Forschungszentrum auf dem Gebiet Photovoltaik. �ber 70 Mitarbeiter sind in den Bereichen Elektrotechnik, Maschinenbau, Physik und Chemie angestellt. Einen gr��eren Teil der Mitarbeiter bilden Studenten und ehemalige Studenten. Zu den Aufgabenfeldern der Adensis GmbH geh�ren das Durchf�hren von Tests und Analysen, Entwicklung neuer Technologien und Produkte sowie Kraftwerksbau. \\
Mein 20-Wochiges Praktikum wurde in der Abteilung Kraftwerkstechnik der Firma Adensis absolviert. Das Pflichtpraktikum war in zwei Hauptgebiete geteilt. Im Oktober habe ich mich mit einem Batterie-Management-System (BMS), welches in der Adensis entwickelt wurde, besch�ftigt. Meine Aufgabe bestand in der Entwicklung eines Algorithmus f�r das BMS. Dieser Algorithmus soll die gemessenen Daten der jeweiligen Batteriezellen in bestimmte Reihenfolgen sortieren. Damit die Daten sp�ter in richtiger Reihenfolge visualisiert werden k�nnen.\\
Ab November war meine Aufgabe die Visualisierung von gesendeten Daten aus einer Kraftwerksanlage. Dieses soll mittels einer Website, mit jeweiligen grafischen Mitteln realisiert werden. Es wurde erforderlich, dass die meisten Daten in Diagramme dargestellt werden. Das Endprodukt soll vor allem den Angestellten dienen, um die Daten eines Kraftwerkes zu analysieren.\\ }


\begin{figure}[htbp]
  \centering
     \includegraphics[width=0.7\textwidth]{images/adensis_logo.jpg}
  \caption{Logo Adensis}
  \label{fig:Logo Adensis}
\end{figure}