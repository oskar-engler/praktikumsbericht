\chapter*{Einleitung}\thispagestyle{fancy}\markboth{Einleitung}{1}
\addcontentsline{toc}{chapter}{Einleitung}


\large {Die Firma Belectric GmbH wurde im 2001 gegr�ndet und hat sein Standort im Kolitzheim. Seit dem hat �ber 1,5 GWp Solarleistung weltweit installiert. Sie wurde somit eine Weltmarktf�hrer in den Bereich Installation von Freifl�chensolarkraftwerken. Es werden neue und innovative Technologien bei der Installation umgesetzt. Weltweit sind �ber 1600 besch�ftigte Menschen, die in Bereichen von Wartung und Anlagebau bis zu Forschen und Entwickeln beim Belectric arbeiten. \\

\begin{figure}[htbp]
  \centering
     \includegraphics[width=0.7\textwidth]{images/Belectric-Logo.png}
  \caption{Logo Belectric}
  \label{fig:Logo Belectric}
\end{figure}

Die Firma Adensis GmbH, mit dem Standort im Dresden, geh�rt zu den Entwicklungs- und Forschungsgruppen der Belectric GmbH. Sie wurde im 2006 gegr�ndet und seit dem betreibt ein Forschungszentrum f�r den Gebiet Photovoltaik. �ber 70 Mitarbeitern sind in den Bereichen Elektrotechnik, Maschinenbau, Physik und Chemie angestellt. Einen gr��eren Teil der Mitarbeiter bilden Studenten und ehemalige Studenten.  Zu den Aufgabenfeldern der Adensis GmbH geh�rt durchf�hren von Testen und Analysen, Entwicklung neuer Technologien und Produkte sowie Kraftwerksbau. \\
Meine 20-Wochige Praktikum wurde in der Abteilung Kraftwerkstechnik der Firma Adensis absolviert. Das Pflichtpraktikum war auf zwei Hauptgebieten geteilt. Im Oktober habe ich mich mit einen Batterie-Management-System, der im Adensis entwickelt war, besch�ftigt. Meine Aufgabe war einen Algorithmus f�r die BMS  zu entwickeln. Dieses Algorithmus soll die gemessene Daten der jeweiligen Batteriezellen nach Reihenfolge sortieren. Damit die Daten sp�ter in Visualisation im richtige Reihenfolge angezeigt werden k�nnen.\\
Ab November war meine Aufgabe die Visualisierung von gesendeten Daten aus einer Kraftwerksanlage. Das soll mittels einer Website, mit jeweiligen Grafischen Mitteln realisiert werden. Es wurde erforderlich, dass die meisten Daten in Diagrammen dargestellt werden. Das Endprodukt soll vor allem den Angestellten dienen um die Daten einer Kraftwerk zu analysieren.\\ }


\begin{figure}[htbp]
  \centering
     \includegraphics[width=0.7\textwidth]{images/adensis_logo.jpg}
  \caption{Logo Adensis}
  \label{fig:Logo Adensis}
\end{figure}