\documentclass[a4paper,11pt, titlepage]{scrreprt}
%basic page

\linespread{1,5}				%Zeilenabstand 1,5mm
\parindent0cm						%Absatz wird nicht reingerückt

\usepackage[utf8]{inputenc}
\usepackage{graphicx}				%Zum Einbinden von externen Grafiken (jpeg,pdf)
\usepackage{a4wide}					%Ausnutzen der ganzen Blattbreite
\usepackage{fancyhdr} 			%Package für Kopf- und Fusszeile
\usepackage[ngerman]{babel}	%Rechtschreibung
\usepackage{amsmath}				%Paket für mathematische Formeln
\usepackage{amsfonts}				%zusätzliche mathematische Zeichen
\usepackage{amssymb}				%zusätzliche mathematische Symbole
\usepackage[labelfont=bf]{caption}  %"Abbildung" fett geschrieben
\usepackage{subfigure} 			%Bilder da plazieren wo sie auch im Latex-Code stehen
\usepackage{graphicx}
\usepackage{eso-pic,picture}
\usepackage[absolute]{textpos}
\usepackage{hyphenat}
\usepackage[onehalfspacing]{setspace}
\usepackage{multirow}
\usepackage{array}
\usepackage{caption}
\usepackage{url}
\usepackage{chngcntr}
\usepackage{tikz}
\counterwithin{figure}{section}
\usepackage[a4paper, left=3cm, right=2.5cm, top=2.0cm, bottom=2.0cm,footskip=1.5cm, includeheadfoot] {geometry}  % manuelles Seitenlayout
\usepackage{bibgerm}        % Einfügen des Literaturverzeichnisses
\usepackage{eurosym}        % Einfügen des € Zeichens
%\usepackage{lscape}        % Text einer Seite wird im quer dargestellt
%\usepackage[a4paper]{geometry}       % Querformat einer Seite
\usepackage{pdfpages}       % Einfügen von ganzen PDF-Seiten
\usepackage{array}
\usepackage{hyperref}				%Einfügen von Weblinks
%================================================================================

%% definieren einer neuen Tabelle mit automatischen Umbruch
\newcolumntype{C}[1]{>{\centering\arraybackslash}m{#1}}%m-vertikal zentriert, b bottom, p top; allg: zur Definition einer festen Spaltenbreite
\newcolumntype{L}[1]{>{\raggedright\arraybackslash}p{#1}} % linksbündig mit Breitenangabe


%================================================================================


\begin{document}\thispagestyle{empty}

%Titelseite
%--------------------------------------------------------------------------------

\thispagestyle{empty} 		%Keine Kopf- und Fusszeilen, keine Seitenzahl
\begin{figure}
\includegraphics[scale=0.3]{images/HTW-Logo}
\end{figure}

\hrule
\vspace{0.2 cm}
\textbf{Fakultät Elektrotechnik}
\vspace{0.1 cm}
\hrule
\vspace{0.2 cm}
Studiengang: Mechatronik/Fahrzeugmechatronik 

\begin{center}
\vspace{2 cm}
\huge\textbf{PRAKTIKUMSBERICHT}
\vspace{2 cm}
\end{center}

\begin{tabular}{lL{7cm}}
\Large\textbf{Thema:} & \Large{Praktikumsbericht}\\
\Large\textbf{Bearbeiter:} & \Large{Oskar Engler}\\
\Large\textbf{Matrikelnummer:} & \Large{34431}\\
\Large\textbf{Bearbeitungszeitraum:} & \Large{01.10.14 bis 28.02.15}\\
\Large\textbf{Ort, Datum der Abgabe:} & \Large{Dresden, DATUM}\\
\Large\textbf{Betreuer:} & \Large{Dipl. Ing. Lars Mademann}\\
\Large\textbf{Verantwortliche. Hochschullehrer:} \hspace{1 cm} & \Large{Prof. Dr.-Ing. Ralf Boden}\\
\vspace{1 cm}\\
\large{Textseiten:} & \large{xx}\\
\large{Anlagen:} & \large{yy}\\
\large{Anhänge:} & \large{zz}\\
\end{tabular}


\linespread{1.5}

%Aufgabenstellung
%---------------------------------------------------------------------------------
\fancyhf{}

\thispagestyle{fancy}
\fancyhead[L]{Oskar Engler}
%Kopfzeile mittig
\fancyhead[C]{Matr-Nr.: 34431}
\fancyhead[R]{HTW-Dresden}
%\renewcommand{\headrulewidth}{0.5pt}

\pagebreak 

%Hauptteil
%---------------------------------------------------------------------------------
%Kapitel sind für eine bessere Übersicht in eigenen Dateien und werden hier aufgerufen:

\pagenumbering{Roman}
\chapter*{Sperrvermerk}\thispagestyle{fancy}\markboth{Sperrvermerk}{1}
\addcontentsline{toc}{chapter}{Sperrvermerk}

Diese Praktikumsbericht enthält vertrauliche Informationen, die der Geheimhaltung unterliegen. Sie dürfen nur für die interne Verwendung und zur Kontrolle durch den verantwortlichen Hochschullehrer genutzt werden. Eine, auch nur teilweise, Veröffentlichung der Belegarbeit darf nur mit Zustimmung der BELECTRIC GmbH, Zweigstelle Dresden, Industriestraße 65, 01129 Dresden erfolgen.
\\ 
\\
\\
{\LARGE Dresden, DATUM}\\

%Inhaltsverzeichnis plotten
%---------------------------------------------------------------------------------
\addtocontents{toc}{\protect\thispagestyle{fancy}}
\thispagestyle{fancy}
\tableofcontents
\thispagestyle{fancy}

%Kopfzeile
%-------------------------------------------------------------------------------- \fancyhf{} %alle Kopf- und Fußzeilenfelder bereinigen
\pagestyle{fancy} %eigener Seitenstil

%Kopfzeile links bzw. innen
\fancyhead[L]{Oskar Engler}
%Kopfzeile mittig
\fancyhead[C]{Matr-Nr.: 34431}
%\facnyhead[R]{\chaptermark}
\renewcommand{\headrulewidth}{0.5pt}				%obere Linie

%Fußzeile
%--------------------------------------------------------------------------------
\fancyfoot[R]{\thepage} 							%Seitennummer
\fancyfoot[L]{\nouppercase{\leftmark}}							%Kapitelbezeichnung Position
\renewcommand{\chaptermark}[1]{\markboth{\thechapter.~\chaptername: #1}{}}	%Kapitelbezeichnung 

\setcounter{page}{2}
\pagenumbering{Roman}
% Abkürzungsverzeichnis
%---------------------------------------------------------------------------------

\chapter*{Abkürzungsverzeichnis}\thispagestyle{fancy}\markboth{Abkürzungsverzeichnis}{Abkürzungsverzeichnis}
\addcontentsline{toc}{chapter}{Abkürzungsverzeichnis}

\begin{tabbing}
\ \= \textbf{Abkürzung} \hspace{0.5cm} \= \textbf{Beschreibung} \kill
\\
\> AG \>  Abgas \\
\> AP \>  Arbeitspaket \\
\> BS \>  Brennstoff \\
\> BK \>  Brennkammer \\
\> BKE \>  Brennkammerende \\
\> etc. \>  et cetera \\
\> f. \>  folgende \\
\> ff. \>  fortfolgende \\
\> u. a. \>  unter anderem \\
\> usw. \>  und so weiter \\
\> z. B. \>  zum Beispiel \\
\> ZWSF \>  Zirkulierende Wirbelschichtfeuerung \\
\end{tabbing}

% Symbolverzeichnis
%---------------------------------------------------------------------------------

\chapter*{Symbolverzeichnis}\thispagestyle{fancy}\markboth{Symbolverzeichnis}{Symbolverzeichnis}
\addcontentsline{toc}{chapter}{Symbolverzeichnis}

\section*{Formelzeichen}

\begin{tabbing}
\ \= \textbf{Symbol} \hspace{0.5cm} \= \textbf{Einheit} \hspace{1.5cm} \= \kill
\> $A$ \> $\mathrm{m^{2}}$ \> Fläche \\
\> $a$ \> $\mathrm{MPa}$ \> Kohäsionsdruck \\
\end{tabbing}

\section*{Griechische Symbole}

\begin{tabbing}
\ \= \textbf{Symbol} \hspace{0.5cm} \= \textbf{Einheit} \hspace{1.5cm} \= \kill
\> $\alpha$ \> $\mathrm{W\,/\,(m^{2} \cdot K)}$ \> Wärmeübergangskoeffizient\\
\> $\gamma$ \> - \> Aktivitätskoeffizient für die Flüssigphase \\
\end{tabbing}
% Abbildungsverzeichnis
%---------------------------------------------------------------------------------
\setcounter{lofdepth}{-1}
\listoffigures
\setcounter{lofdepth}{-1}
\setcounter{page}{1}
\pagenumbering{arabic}
{\inputencoding{latin1}\chapter*{Einleitung}\thispagestyle{fancy}\markboth{Einleitung}{}
\addcontentsline{toc}{chapter}{Einleitung}

Die Firma Belectric GmbH wurde im 2001 gegr�ndet und hat sein Standort im Kolitzheim. Seit dem hat �ber 1,5 GWp Solarleistung weltweit installiert. Sie wurde somit eine Weltmarktf�hrer in den Bereich Installation von Freifl�chensolarkraftwerken. Es werden neue und innovative Technologien bei der Installation umgesetzt. Weltweit sind �ber 1600 besch�ftigte Menschen, die in Bereichen von Wartung und Anlagebau bis zu Forschen und Entwickeln beim Belectric arbeiten. 
Die Firma Adensis GmbH, mit dem Standort im Dresden, geh�rt zu den Entwicklungs- und Forschungsgruppen der Belectric GmbH. Sie wurde im 2006 gegr�ndet und seit dem betreibt ein Forschungszentrum f�r den Gebiet Photovoltaik. �ber 70 Mitarbeitern sind in den Bereichen Elektrotechnik, Maschinenbau, Physik und Chemie angestellt. Einen gr��eren Teil der Mitarbeiter bilden Studenten und ehemalige Studenten.  Zu den Aufgabenfeldern der Adensis GmbH geh�rt Durchf�hren von Testen und Analysen, Entwicklung neuer Technologien und Produkte sowie Kraftwerksbau. Meine 20-Wochige Praktikum wurde in den Abteilung Kraftwerkstechnik der Firma Adensis absolviert. 
Das Pflichtpraktikum war auf zwei Hauptthemen geteilt. Im Oktober habe ich mich mit einen Batterie-Management-System, der im Adensis entwickelt war,  besch�ftigt. Meine Aufgabe war einen Algorithmus f�r die B&R Steuerung entwickeln sollen. 
Ab November war ich mit Visualisierung von gesendeten Daten aus einer Kraftwerksanlage besch�ftigt. Das soll mittels einer Website, mit jeweiligen Grafischen Mitteln realisiert werden.}
{\inputencoding{latin1}\chapter{Entwicklung der BMS-Algorithmus}\thispagestyle{fancy}


\section{Einf�hrung in die Batterie Management System}
\large{(Was ist das?, wozu nutzt man?, die Einweisungen)}


\section{Erstellung eines BMS-Planes}
\large{}

%Meine schritte, was ich gemacht habe, erst einen Plan erstellt - damit ich und andere einen sch�nen �bersicht �ber die Verteilung von BMS auf der jeweiligen Tr�gen haben..  Die Excel Tabellen erstellt, einen System gefunden, sehr viel Berechnen, logische Denken, Spannungswerten, Temperatur, Leitwert, Error

\section{BMS Algorithmus in C entwickeln}
\large{(Mit C Sprache einen Algorithmus entwickelt, es war meine Wahl, Einen Algorithmus in den ganzen System gefunden, die jeweiligen Trogverbund, Tr�ge, Batteriezellen)}


\section{Umwandlung in den \glqq Strukturierten Text\grqq \hspace{1mm}ST}
\large{(Damit es Johann implementieren kann, muss ich es in \textbf{ST} umwandeln, \textbf{ST} Syntax lernen, Die Implementierung)}


\section{Resultate}
\large{(Anwendung f�r die B\&R Steuerung, Screenshot von Johann)}}
{\inputencoding{latin1}\chapter{Web-Visualisierung}\thispagestyle{fancy}

\section{Zielstellung}
\large
{Ab November entstand ein neues Projekt mit dem Ziel einer Datenvisualisierung von einem Kraftwerk. Die Visualisierung erfolgt dabei in Form von Diagrammen und Tabellen auf einer eigens eingerichteten Website. Am Anfang wurde die Zielstellung des Projektes in einer Reihe von Besprechungen erarbeitet. In diesen Besprechungen wurden die jeweiligen Aufgaben verteilt. Die Abteilung Eingebettete Systeme sollte einen Webserver mit einer SQL-Datenbank bereitstellen, auf der die Live-Daten des Batteriekraftwerkes bereitgestellt werden. Die Aufgabe war die Visualisierung der Daten. Die visualisierten Daten sollen auf einer Website dargestellt werden. \\
Die Daten aus dem Kraftwerk sollen �bersichtlich und mittels Diagrammen visualisiert werden. Die Eignung der Daten zur Visualisierung wurde mit dem Betreuer abgestimmt. Es wurde ebenfalls besprochen, welche Mitteln f�r die Website Erstellung geeignet sind. Dazu geh�ren Bootstrap Frameworks\footnote{Bootstrap ist [...] Framework mit HTML, CSS und JS f�r die Entwicklung von anpassungsf�higen Projekten f�r das moderne Web.\cite{boots}}, jQuery({weil: ,,Ohne JavaScript-Bibliothek m�ssen Entwickler h�ufig viele Zeilen Code schreiben, um den DOM-Baum (Document Object Model) zu durchlaufen und einzelne Teile der Struktur eines HTML-Dokuments zu finden. Mit jQuery steht den Entwicklern ein robuster und effizienter Selektionsmechanismus zur Verf�gung, der es einfach macht, genau den Teil eines Dokuments abzurufen, den Sie untersuchen oder bearbeiten wollen"'.}\cite[S. 7-8]{jquery}) und dazugeh�rige Software f�r die Bearbeitung.}


\section{Konzipierung}
\large{Es soll ein Konzept f�r die Website entwickelt werden. Es wird ein Konzept mit dem Grafikprogramm GIMP\footnote{GIMP ist ein freies, sehr leistungsf�higes Photo- und Bildbearbeitungsprogramm.\cite{gimp}}. erstellt, wo die erste Konzepte f�r die Anmeldung-Seite (siehe Abbildung \ref{login_seite}) und f�r die Seite, wo die jeweiligen L�nder und Anlagen angezeigt werden (siehe Abbildung \ref{Anlagenblocke}), gemacht wurden. Damit die Auswahl der jeweiligen L�nder noch �bersichtlicher wird, ist unser Team auf der Idee gekommen, dass wir eine gro�e Karte in die Seite implementieren werden, wo der Benutzer aus der jeweiligen Region/Land ein Kraftwerk ausw�hlen kann. Es wurden auch erste Konzepte f�r die Tabellen, Fehlermeldungen und Diagrammen gemacht.}

\vspace{0,5cm}

\begin{figure}[htbp]
	\centering
		\includegraphics[width=0.7\textwidth]{images/start_login.jpg}
		\caption{Login-Seite}
		\label{login_seite}
\end{figure}

\vspace{0,5cm}

\begin{figure}[htbp]
  \centering
     \subfigure[Vor dem Klick]{\includegraphics[width=0.4\textwidth]{images/Liste_Kontinente.PNG}}
			\hspace{1cm}
		 \subfigure[Nach dem Klick]{\includegraphics[width=0.4\textwidth]{images/Liste_Anlagenblocke.PNG}}
	\caption{Liste mit L�nder und Anlagen}
  \label{Anlagenblocke}
\end{figure}


\section{Website - Entwicklung}
\large{N�chste Aufgabe war, die entwickelte Konzepte mit statischem HTML, CSS und Javascript umzusetzen. Unser Team hat abgesprochen, dass wir f�r die Website Entwicklung das Bootstrap Frameworks benutzen werden. Das hat sehr viele Vorteile, als beim Null zu starten. Erstens ist das Bootstrap auch f�r kommerzielle Benutzung kostenlos, zweitens es beschleunigt die Arbeit, weil es sozusagen schon die CSS-Programmierung enth�lt und wir nutzen nur die vorkonfiguriertes CSS Style f�r die jeweiligen HTML Elementen und drittens Bootstrap bietet sehr kompatible und responsive Design, das hei�t, dass die Seite auch f�r alle Browsers sowie Handys und Tablets optimiert ist. F�r die Programmierung wurde die Software Sublime Text genutzt.\\
Das erste Konzept der Login-Seite war mit Bootstrap gemacht (siehe Abbildung \ref{login_seite_boot}), eine einfache HTML Seite mit Belectric Logo, Anmeldungsfelder und einen Button 'Sign In'.\\}

\begin{figure}[htbp]
	\centering
		\includegraphics[width=0.8\textwidth]{images/login_seite_boot.png}
		\caption{Login-Seite mit Bootstrap}
		\label{login_seite_boot}
\end{figure}

\large{Danach wurde ein Konzept f�r die Auswahl-Liste mittels HTML und Bootstrap realisiert. Mit dem Bootstrap kann man schnell die Liste erstellen und auch die Breite jedes Listenelements mit mit HTML/CSS definieren. Daf�r wird das Grid-System, welches von Bootstrap entwickelt wurde verwendet. Der Grid-System stellt uns eine Zw�lfspaltigen-Layout, welche die Breite der jeweiligen HTML Elementen definiert. Dazu gibt es mehrere CSS-Klassen, welche f�r die Breite des kleinen oder gro�en Monitors, Handy-Display oder Tablet zugeordnet sind. Man kann damit eine Seite responsive machen, weil man kann die Anzeige an allen verschiedenen Monitors-Gr��en einstellen. Es funktioniert so, dass die Breite einen erstellten \textbf{div} mit \textbf{collumns} mit unterschiedlicher Gr��e einstellbar ist. Mit der maximalen Breite 12 wird so geschrieben: \textbf{col-lg-12} f�r gro�e Monitoren bzw. \textbf{col-sm-12} f�r kleine Displays. Das, was in diesem \textbf{div} dargestellt wird, wird sich auch �ber die ganze Breite der Seite strecken.\\
In unserem Fall, war \textbf{col-lg-8} f�r die Kontinente eingestellt und einen \textbf{col-lg-offset-1} f�r die untergeordnete Liste mit L�nder (siehe Abbildung \ref{html_list_kontinente}), damit die um einen col-1 nach rechts verschoben werden. Mit diesem Fortgang sind auch weitere Liste entstanden.}

\begin{figure}[htbp]
	\centering
		    \subfigure[HTML - Quellcode]{\includegraphics[width=\linewidth]{images/html_list_kontinente.png}}
				\subfigure[Ansicht im Browser]{\includegraphics[width=0.6\textwidth]{images/List_Kontinente_amerika.png}}
		\caption{Liste mit Amerika}
		\label{html_list_kontinente}
\end{figure}

\large{Wichtig war, dass die wichtige Informationen nach dem Anklicken eines Kraftwerkes angezeigt werden. Das wurde mit einer HTML-Tabelle umgesetzt. Man muss die Zeilen und Spalten definieren und alles in ein definiertes Feld (div) einbetten. Man muss darauf achten, dass die jeweiligen Tabellen den jeweiligen Listenelementen untergeordnet sind. Die HTML \textbf{table} ist auch von den Bootstrap stilisiert und verh�lt sich responsive. Damit es mit Bootstrap CSS funktioniert, muss man in den \textbf{class} derjenige Tabellen die richtigen Class-Namen f�r das Verhalten dieser HTML Element eingeben. Es wurden die Klassen [\textbf{class="'table-responsive table table-bordered"'}] benutzt (siehe Abbildung \ref{List_Kontinente_Europa_Germany}). �ber CSS wurden noch Farben zu Unterscheidung eingestellt und Effekte wie, wenn man �ber die Listenelementen mit dem Cursor �bergeht, bekommt der Element eine andere Farbe.}

\begin{figure}[htbp]
	\centering
		\includegraphics[width=1\textwidth]{images/List_Kontinente_Europa_Germany.png}
		\caption{Liste mit Deutschlands Kraftwerken}
		\label{List_Kontinente_Europa_Germany}
\end{figure}

\large{Damit die jeweiligen Listenelementen nach dem Klick immer noch auf einer Seite ge�ffnet werden, wurde ein 
jQuery Skript entwickelt. Das Ziel war, dass beim Anklicken das untergeordnete Listenelement bzw. Tabelle nach unten rutscht. Die jQuery biete eine solche Funktion: \textbf{slideToggle}, welche der Absicht entspricht. Man muss definieren, welche Elemente dieser Funktion vornehmen sollen (Siehe Abbildung \ref{jQuery_america_slide}). Deswegen m�ssen die HTML Elemente mit IDs oder Klassen verbunden sein. 

\begin{figure}[htbp]
	\centering
		\includegraphics[width=0.5\textwidth]{images/jQuery_america_slide.png}
		\caption{jQuery Skript f�r �ffnen von Amerika Kontinent Liste}
		\label{jQuery_america_slide}
\end{figure}

% Version before sb-admin-2 im Anhang includen und hier referieren
%-----------------------------------------------------------------

\large{Ein festes Design bzw. Layout wurde gew�nscht um die Benutzeroberfl�che angenehm zu gestalten, haben wir uns entscheidet, dass die Seite mit dem \textbf{sb-admin-2} umgebaut wird. Es handelt sich um einen Template\footnote{Eine Schablone oder Muster mit vorgemachten Design und Layout eine Seite.} f�r Bootstrap. Es hat bereits Layout-Elemente wie eine Navigation-leiste und einem responsive Seitenmen�. Wir nutzen dies f�r das Umbauen der Seite. Die Login Seite ist geblieben und die anderen Seiten wurden teilweise in das Template implementiert.\\
Als erste wurde die Idee mit den Karten und der Auswahl von Kraftwerken aus jeweiligen L�nder mit Pointers umgesetzt. Dazu wurde eine Javascript-Bibliothek und die Open-Source Karten ben�tigt. F�r diesen Fall war die Javascript Bibliothek \textbf{Leaflet} die L�sung. Es ist leicht anwendbar und auch responsive. Damit es funktionieren kann, braucht man eine Karten-Ebene und die Lagekoordinaten. Dann wird ein Skript zum darstellen dieser Karte geschrieben (Siehe Abbildung \ref{leaflet_map}). Dieses Skript wurde in die Karte mit div definiert und damit konnte die Karte angezeigt werden. In Abbildung \ref{new_karte_auswahl} ist die Ansicht im Browser zu erkennen.}

\hspace{2cm}

\begin{figure}[htbp]
	\centering
		\includegraphics[width=0.99\textwidth]{images/leaflet_map.png}
		\caption{jQuery Leaflet Karte mit Pointer Beispielen}
		\label{leaflet_map}
\end{figure}

\begin{figure}[htbp]
	\centering
		\includegraphics[width=0.9\textwidth]{images/new_karte_auswahl.png}
		\caption{Die Karte im Seite mit neuen Layout}
		\label{new_karte_auswahl}
\end{figure}

\large{Nach der Kraftwerksauswahl erfolgt eine Weiterleitung zum Dashboard, wo sich alle wichtige Informationen bzw. Daten befinden. Die Dashboard soll so aussehen, dass der Benutzer die Statusmeldungen und paar Diagrammen auf den ersten Blick sieht. Das erfolgt mittels Panels, Statusleisten, Charts (Diagrammen) und andere Elementen. Einige HTML Elemente sind bereit mit sb-admin-2 CSS-stilisiert. Anfangs wurde mit Tabellen gearbeitet (siehe Abbildung \ref{first_sb_admin}), wo die wichtige Kenndaten dargestellt wurden. Es wurde ein neues Konzept entwickelt und nach der Einigung mit dem Betreuer wurde es in das Bootstrap umgewandelt (siehe Abbildung \ref{erste_dashboard}).\\

\begin{figure}[htbp]
	\centering
		\includegraphics[width=1\textwidth]{images/first_sb_admin.png}
		\caption{Die erste Umsetzung mit Tabellen als Beispiel}
		\label{first_sb_admin}
\end{figure}

\begin{figure}[htbp]
	\centering
		\includegraphics[width=1\textwidth]{images/erste_dashboard.png}
		\caption{Ein Teil der Test-Dashboard ohne Navigationsleisten}
		\label{erste_dashboard}
\end{figure}

\large{Im linken Men� sollen die jeweiligen Module des Kraftwerkes stehen. Beim Anklicken wird die Information, Status-Meldungen, Diagrammen bzw. Tabellen zu dem Modul angezeigt. Das Men� wird mit Bootstrap und mit HTML mit CSS Klassen aufgebaut.\\ 
Es wurde noch die \textbf{collapse} Funktion eingebaut, das beim Anklicken einer zusammengesetzten Gruppe, zum Beispiel: EBU Master und EBU Slave als EBU, wird das nach unten rutschen und die zwei EBUs anzeigen. Damit es funktioniert, muss man zu den Men�-Elementen im HTML �ber die Klassen die Funktion einbinden (siehe Abbildung \ref{links_menu_modules}). Die Funktion selbst ist mit jQuery geschrieben und ist schon im sb-admin-2 eingef�hrt.\\ }

\begin{figure}[htbp]
	\centering
				\subfigure[HTML - Quellcode Linke Men�]{\includegraphics[width=1\textwidth]{images/links_menu_modules.png}}
				\subfigure[im Browser: EBU Master aktiv]{\includegraphics[width=0.22\textwidth]{images/links_menu_modules_browser.png}}
		\caption{Modulen-Men� links}
		\label{links_menu_modules}
\end{figure}

\pagebreak


\section{Die SQL-Datenbank}
\large{Die postgreSQL\footnote{PostgreSQL [...] ist ein freies, objektrelationales Datenbankmanagementsystem (ORDBMS).} ist ein essenzieller Bestandteil des Projektes. In dieser Datenbank liegen die Messwerte und Kraftwerkskenndaten vor. Es gibt eine Tabelle f�r die Meta-Daten und dann eine Tabellen f�r die gesendeten Daten eines Kraftwerkes. Unser Team hat abgesprochen, wie die Meta-Datenbank\footnote{Meta-Datenbank dient in unseren Fall als Datenbank mit Kenndaten und allgemeine Information, wie zum Beispiel: Name, Lage, Baujahr usw. eines Kraftwerks} aussehen wird. Die n�chste Aufgabe war dann die Datenbank erstellen. Das kann auf der Kommandozeile oder einem Web-Frontend z.B. Adminer geschehen, in dem SQL-Abfragen abgesetzt werden k�nnen.(siehe Abbildung \ref{sql_dump}).

\begin{figure}[htbp]
	\centering
		\includegraphics[width=1\textwidth,height=10cm]{images/sql_dump.png}
		\caption{Ein Teil der implementiertes SQL Quellcode}
		\label{sql_dump}
\end{figure}
\pagebreak

Dort werden die Tabellen der Datenbank mit allen Spalten und Datentypen definiert. Im unseren Fall wurden in der Datenbank Tabelle: continents, countries, countrycoordinates f�r die Karte, ebupowerplants mit Namen und Lagen, ebupowerplantsetups, wo die Modulen definiert werden, locations, modules, modulesname mit Module-Namen, temperature und users (siehe Abbildung \ref{adminer_meta}). Die Tabellen wurden teilweise mit Daten gef�llt. Die Ausgabe von Daten an die Diagrammen wird sp�ter mit Hilfe von PHP-Funktion realisiert.}
%include anhang sql dump

\begin{figure}[htbp]
	\centering
		\includegraphics[width=1\textwidth]{images/adminer_meta.png}
		\caption{Ansicht der Metadatenbank im Adminer}
		\label{adminer_meta}
\end{figure}

\pagebreak

\section{Datenvisualisierung - Charts}
\large{Der n�chste Schwerpunkt war eine Javascript oder jQuery Bibliothek zu finden, welche die Daten in geeigneten Diagrammen visualisieren und dynamisch Inhalte nachladen kann. Die M�glichkeiten wie 'Export Diagramm als Bild', Timeline\footnote{Ausw�hlen einen Zeitabschnitt}, direktes Datum-Auswahl, Zoom, gleichzeitige Vergleichen von mehreren Daten waren die Vergleichspunkte. Nach dem Untersuchen von mehreren freien oder auch kostenpflichtigen Bibliotheken war mit dem Betreuer abgesprochen, dass auf der �bersichtsseite und den Seiten der Module eine einfache Bibliothek: \textbf{dyGraphs} benutzt werden soll. Die \textbf{dyGraphs} sind einfach, responsive, zoom-f�hig und k�nnen Daten im Format von CSV\footnote{Comma Separated Values (CSV) ist ein Textdateiformat, mit dem Sie Daten aus einem Datenbank- oder Tabellendokument anwendungs�bergreifend austauschen k�nnen.\cite{csv}}visualisieren (siehe Abbildung \ref{dygraphs_test}). Es war noch abgesprochen, dass es einen Link zur neuen Fenster mit Diagramm-Bibliothek \textbf{Amstockcharts} erstellen soll. Die \textbf{Amstockcharts} sind das Vergleichen mehrerer Datens�tze gut geeignet. Man kann auch die Diagrammen exportieren und man hat viele M�glichkeiten, welchen Zeitabschnitt dargestellt sein soll (siehe Abbildung \ref{amstockcharts_test}). 

\begin{figure}[htbp]
	\centering
		\includegraphics[width=0.9\textwidth]{images/dygraphs_test.png}
		\caption{dyGraphs Beispiel mit Beispielsdaten}
		\label{dygraphs_test}
\end{figure}

\begin{figure}[htbp]
	\centering
		\includegraphics[width=1\textwidth]{images/amstockcharts_test.png}
		\caption{Amstockcharts Beispiel mit Beispielsdaten}
		\label{amstockcharts_test}
\end{figure}

\large{Die Diagramme werden von Javascript-Bibliotheken bereitgestellt und m�ssen auch via Javascript initialisiert und konfiguriert werden. So ist erforderlich, das die Einstellungen f�r Achsen, Beschriftungen und Export-Optionen mit Hilfe eines Skriptes im Quellcode definiert werden m�ssen. Verschiedene Bibliotheken erfordern verschiedene Syntax. Es wurden
zwei unterschiedliche Bibliotheken benutzt, dass hei�t es wurde mit zwei unterschiedlichen Syntaxen die Scripte geschrieben. \\
Bei der \textbf{dyGraphs} ist zu beachten, dass die Daten im Form von CSV sind oder in reines HTML gespeichert sind. In den Optionen kann man Titel, Achsenbeschreibung, Legende, Position der Beschriftung definieren und ob man dann im Browser gleitende Mittelwert der Diagramm einstellen kann (siehe Abbildung \ref{dyGraphs_JS}). Das Diagramm wird in einen definierten \textbf{div} angezeigt. Der \textbf{div} soll eine konkrete Gr��e haben. Wenn man der Diagramm responsive machen will, muss man in den CSS-Style die Gr��e statt in Pixels in Prozent eingeben.\\

\begin{figure}[htbp]
	\centering
		\includegraphics[width=0.8\textwidth]{images/dyGraphs_JS.png}
		\caption{dyGraphs Beispiel-Diagramm}
		\label{dyGraphs_JS}
\end{figure}

Die Amcharts sind komplexere, gr��ere und reine Javascript Charts-Bibliothek, die man unter bestimmten Bedingungen auch fre benutzen kann. Es gibt auch mehrere M�glichkeiten, wie der Diagramm dargestellt werden k�nnen. Man muss, wie beim dyGraphs, die Quelle der Daten, Beschriftungen, Achseneinstellungen und andere einstellen (siehe Abbildung \ref{amstockcharts_JS}).}
%include JS von charts

\begin{figure}[htbp]
	\centering
				\includegraphics[width=0.95\textwidth]{images/amstockcharts_JS.png}
		\caption{Amstockcharts Beispiel-Diagramm}
		\label{amstockcharts_JS}
\end{figure}
\pagebreak

\section{PHP}
\large{Der n�chste Schwerpunkt war die dynamische Umsetzung in PHP\footnote{ist eine Skriptsprache mit einer an C und Perl angelehnten Syntax, die haupts�chlich zur Erstellung dynamischer Webseiten oder Webanwendungen verwendet wird. [..] PHP zeichnet sich durch breite Datenbankunterst�tzung und Internet-Protokolleinbindung}. Es wurde ben�tigt, weil die Website dynamisch sein soll. Die bereits erstellte HMTL Seite wurde mit PHP umgesetzt. Nach der Bearbeitung wird es als \textbf{.php} Datei gespeichert und auf den Server hingelegt. Der Server hat einen FTP-Zugang. �ber den kann man die Dateien hochladen und dann die Seite �ber eine IP-Adresse besuchen. Die PHP Skript-Sprache ist eine serverseitige Sprache, das hei�t, dass der Code auf den Server verarbeitet wird. In der PHP-Interpreter wird das Code verarbeitet und erzeugt eine Datei, welche an den Client zur�ckgegeben wird.\\
�ber PHP kann man auf die Datenbank zugreifen und somit die Daten in die Diagramme einpflegen. Erstens muss man sich mit Datenbank verbinden. Das erfolgt mittles \textbf{connect} und in unseren Fall das \textbf{pg\_connect}, weil wir die postgreSQL benutzen. Nach der Verbindung mit SQL kann man mit den Befehlen \textbf{pg\_query} die Daten aus der Datenbank auf eine Variable aufladen. Diese Variable wird dann benutzt, um den Diagramm richtigen Weg zur Daten konfigurieren. \\ 
%includen code von connect und query
Die Daten werden in eine Form gespeichert, welche die Diagramme nicht verarbeiten k�nnen. Das hei�t, man muss die Umwandlung in die Form von JSON oder CSV durchf�hren. Diese sind von Diagrammen akzeptiert. Die Umwandlung in JSON kann man mit \textbf{encode\_json} machen. F�r die CSV Form wurde ein Parser\footnote{Ein Parser ist ein Computerprogramm, das in der Informatik f�r die Zerlegung und Umwandlung einer beliebigen Eingabe in ein f�r die Weiterverarbeitung brauchbares Format zust�ndig ist.} geschrieben (siehe Abbildung \ref{csv_parser}).\\ %siehe include parserCSV 
Die \textbf{pg\_query} wird nicht nur f�r die Diagrammen benutzt sondern auch f�r Men� mit L�ndern, Men� mit Modulen eines Kraftwerkes und andere. �ber diese \textbf{PHP} Funktionen kann man die Meta-Daten benutzen um dynamische Men�s bauen. Wenn zum Beispiel die Kraftwerk in Alt Daber zwei EBUs, zwei Wechselrichter, zwei BMS und einen Pool hat, so wird das auch im Meta-Datenbank gespeichert und auf der Website wird die Men� in diese Konfiguration angezeigt (siehe Abbildung\ref{altdaber_menu_php}). So wird die Seite automatisiert und wenn eine neue Kraftwerk in die SQL hinzugef�gt wird, dann wird es auch an die Seite angezeigt.\\}

\begin{figure}[htbp]
	\centering
				\includegraphics[]{images/csv_parser.png}
		\caption{CSV Parser f�r die Diagramme}
		\label{csv_parser}
\end{figure}

\begin{figure}[htbp]
	\centering
				\includegraphics[width=0.25\textwidth]{images/altdaber_menu_php.png}
		\caption{Men� �ber PHP}
		\label{altdaber_menu_php}
\end{figure}


\section{Test Durchf�hrung}\thispagestyle{fancy}
\large{Der Test besteht daran, dass die Ladezeiten der Website-Diagrammen klein wie m�glich bleiben sollen. Die ersten Ergebnisse zeigen lange Ladezeiten, die sich �ber 1 Sekunde laden. (siehe Abbildung \ref{ebu_sec_test_vor}). Bei mehrere Daten, wie zum Beispiel Frequenzen, Stromwerten und Spannungswerten eines Moduls, k�nnen die Ladezeiten der gesamte Website �ber 20 Sekunden gehen. Damit die Website sich schneller l�dt, wurde die Datenmenge der eingeladenen Werten reduziert und damit die gesamt Ladezeit verringert. Die \textbf{pg\_query} erm�glicht verschiedene Einstellungen, unter welchen sind auch Auswahl der Menge der Daten. Man kann einstellen, dass die \textbf{pg\_query} nur jede zweite oder dritte Zahl sich nehmen soll. Damit wurden die Ladezeiten unter 1 Sekunde gebracht(siehe Abbildung \ref{ebu_sec_test_nach}).}

\begin{figure}[htbp]
	\centering
				\includegraphics[width=1\textwidth]{images/ebu_sec_test_rev.png}
		\caption{Ladezeiten vor dem Reduzieren}
		\label{ebu_sec_test_vor}
\end{figure}

\begin{figure}[htbp]
	\centering
				\includegraphics[width=1\textwidth]{images/ebu_sec_test.png}
		\caption{Ladezeiten nach dem Reduzieren}
		\label{ebu_sec_test_nach}
\end{figure}

\section{Fazit}\thispagestyle{fancy}
\large{Die Website dient jetzt vor allem als Werkzeug zur Datenanalyse, welche an den Server gesendet werden. Die Benutzer k�nnen sich die Daten aus verschiedene Modulen des Kraftwerkes anzeigen lassen. Man kann auch gr��ere Zeitspannen sowie k�rzere Zeit-Intervallen einstellen und dann die Daten vergleichen.}
}
{\inputencoding{latin1}\chapter{Zusammenfassung und Ausblick}\thispagestyle{fancy}

\large{Was habe ich damit geschafft? War ich erfolgreich?}}
{\inputencoding{latin1}% Anhang
%---------------------------------------------------------------------------------

\chapter*{Anhang}\thispagestyle{fancy}\markboth{Anhang}{Anhang}
\addcontentsline{toc}{chapter}{Anhang}

Hier sind Ausdrucke von Quelltexten oder MathCAD-Dokumenten, große Grafiken und Diagramme und Fotoserien gut aufgehoben.}
{\inputencoding{latin1}\input{EidesstattlicheErklarung.tex}}


\end{document}