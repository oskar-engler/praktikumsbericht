\documentclass[a4paper,10pt, titlepage]{scrreprt}
%basic page

\linespread{1.5}				%Zeilenabstand 1,5mm
\parindent0cm						%Absatz wird nicht reingerückt

\usepackage[utf8]{inputenc}
\usepackage{graphicx}				%Zum Einbinden von externen Grafiken (jpeg,pdf)
\usepackage{a4wide}					%Ausnutzen der ganzen Blattbreite
\usepackage{fancyhdr} 			%Package für Kopf- und Fusszeile
\usepackage[ngerman]{babel}	%Rechtschreibung
\usepackage{hyperref}				%Einfügen von Weblinks
\usepackage{amsmath}				%Paket für mathematische Formeln
\usepackage{amsfonts}				%zusätzliche mathematische Zeichen
\usepackage{amssymb}				%zusätzliche mathematische Symbole
\usepackage[labelfont=bf]{caption}  %"Abbildung" fett geschrieben
\usepackage{subfigure} 			%Bilder da plazieren wo sie auch im Latex-Code stehen
\usepackage{graphicx}
\usepackage{eso-pic,picture}
\usepackage[absolute]{textpos}
\usepackage{hyphenat}
\usepackage[onehalfspacing]{setspace}
\usepackage{multirow}
\usepackage{array}
\usepackage{caption}
\usepackage{url}
\usepackage{chngcntr}
\usepackage{tikz}
\counterwithin{figure}{section}
\usepackage[a4paper, left=3.5cm, right=2.5cm, top=2.0cm, bottom=2.0cm,footskip=1.5cm, includeheadfoot] {geometry}  % manuelles Seitenlayout
\usepackage{bibgerm}        % Einfügen des Literaturverzeichnisses
\usepackage{eurosym}        % Einfügen des € Zeichens
%\usepackage{lscape}        % Text einer Seite wird im quer dargestellt
%\usepackage[a4paper]{geometry}       % Querformat einer Seite
\usepackage{pdfpages}       % Einfügen von ganzen PDF-Seiten
\usepackage{array}

%================================================================================

%% definieren einer neuen Tabelle mit automatischen Umbruch
\newcolumntype{C}[1]{>{\centering\arraybackslash}m{#1}}%m-vertikal zentriert, b bottom, p top; allg: zur Definition einer festen Spaltenbreite
\newcolumntype{L}[1]{>{\raggedright\arraybackslash}p{#1}} % linksbündig mit Breitenangabe


%================================================================================


\begin{document}

%Titelseite
%--------------------------------------------------------------------------------
\begin{titlepage}
\thispagestyle{empty} 		%Keine Kopf- und Fusszeilen, keine Seitenzahl
\linespread{1.0}

\begin{figure}
\includegraphics[scale=0.3]{images/HTW-Logo}
\end{figure}

\hrule
\vspace{0.2 cm}
\textbf{Fakultät Elektrotechnik}
\vspace{0.1 cm}
\hrule
\vspace{0.2 cm}
Studiengang: Mechatronik/Fahrzeugmechatronik 

\begin{flushleft}
\vspace{8 cm}
\huge\textbf{Praktikumsbericht}\\ [0.5 cm]
\Large{Bericht zur Praktikum im Adensis GmbH}\\ [1.5 cm]
\Large{Dresden, DATUM}
\end{flushleft}
\pagebreak 

%new page

\thispagestyle{empty} 		%Keine Kopf- und Fusszeilen, keine Seitenzahl
\begin{figure}
\includegraphics[scale=0.3]{images/HTW-Logo}
\end{figure}

\hrule
\vspace{0.2 cm}
\textbf{Fakultät Elektrotechnik}
\vspace{0.1 cm}
\hrule
\vspace{0.2 cm}
Studiengang: Mechatronik/Fahrzeugmechatronik 

\begin{center}
\vspace{2 cm}
\huge\textbf{PRAKTIKUMSBERICHT}
\vspace{2 cm}
\end{center}

\begin{tabular}{lL{7cm}}
\Large\textbf{Thema:} & \Large{Praktikumsbericht}\\
\Large\textbf{Bearbeiter:} & \Large{Oskar Engler}\\
\Large\textbf{Matrikelnummer:} & \Large{34431}\\
\Large\textbf{Bearbeitungszeitraum:} & \Large{01.10.14 bis 28.02.15}\\
\Large\textbf{Ort, Datum der Abgabe:} & \Large{Dresden, DATUM}\\
\Large\textbf{Betreuer:} & \Large{Dipl. Ing. Lars Mademann ??}\\
\Large\textbf{Verantw. Hochschullehrer:} \hspace{1 cm} & \Large{Prof. Dr.-Ing. Ralf Boden}\\
\vspace{1 cm}\\
\large{Textseiten:} & \large{xx}\\
\large{Anlagen:} & \large{yy}\\
\large{Anhänge:} & \large{zz}\\
\end{tabular}

\end{titlepage}
\linespread{1.5}

%Aufgabenstellung
%---------------------------------------------------------------------------------
\fancyhf{}

\thispagestyle{fancy}
\fancyhead[L]{Oskar Engler}
%Kopfzeile mittig
\fancyhead[C]{Matr-Nr.: 34431}
\fancyhead[R]{HTW-Dresden}
%\renewcommand{\headrulewidth}{0.5pt}

\pagebreak 

%Hauptteil
%---------------------------------------------------------------------------------
%Kapitel sind für eine bessere Übersicht in eigenen Dateien und werden hier aufgerufen:

\pagenumbering{Roman}
\chapter*{Sperrvermerk}\thispagestyle{fancy}\markboth{Sperrvermerk}{1}
\addcontentsline{toc}{chapter}{Sperrvermerk}

Diese Praktikumsbericht enthält vertrauliche Informationen, die der Geheimhaltung unterliegen. Sie dürfen nur für die interne Verwendung und zur Kontrolle durch den verantwortlichen Hochschullehrer genutzt werden. Eine, auch nur teilweise, Veröffentlichung der Belegarbeit darf nur mit Zustimmung der BELECTRIC GmbH, Zweigstelle Dresden, Industriestraße 65, 01129 Dresden erfolgen.
\\ 
\\
\\
{\LARGE Dresden, DATUM}\\


%Kopfzeile
%-------------------------------------------------------------------------------- \fancyhf{} %alle Kopf- und Fußzeilenfelder bereinigen
\pagestyle{fancy} %eigener Seitenstil

%Kopfzeile links bzw. innen
\fancyhead[L]{Oskar Engler}
%Kopfzeile mittig
\fancyhead[C]{Matr-Nr.: 34431}
%\facnyhead[R]{\chaptermark}
\renewcommand{\headrulewidth}{0.5pt}				%obere Linie

%Fußzeile
%--------------------------------------------------------------------------------
\fancyfoot[R]{\thepage} 							%Seitennummer
\fancyfoot[L]{\nouppercase{\leftmark}}							%Kapitelbezeichnung Position
\renewcommand{\chaptermark}[1]{\markboth{\thechapter.~\chaptername: #1}{}}	%Kapitelbezeichnung 

\setcounter{page}{1}
\pagenumbering{Roman}
%Inhaltsverzeichnis plotten
%---------------------------------------------------------------------------------
\addtocontents{toc}{\protect\thispagestyle{fancy}}
\thispagestyle{fancy}
\tableofcontents
\thispagestyle{fancy}
% Abkürzungsverzeichnis
%---------------------------------------------------------------------------------

\chapter*{Abkürzungsverzeichnis}\thispagestyle{fancy}\markboth{Abkürzungsverzeichnis}{Abkürzungsverzeichnis}
\addcontentsline{toc}{chapter}{Abkürzungsverzeichnis}

\begin{tabbing}
\ \= \textbf{Abkürzung} \hspace{0.5cm} \= \textbf{Beschreibung} \kill
\\
\> AG \>  Abgas \\
\> AP \>  Arbeitspaket \\
\> BS \>  Brennstoff \\
\> BK \>  Brennkammer \\
\> BKE \>  Brennkammerende \\
\> etc. \>  et cetera \\
\> f. \>  folgende \\
\> ff. \>  fortfolgende \\
\> u. a. \>  unter anderem \\
\> usw. \>  und so weiter \\
\> z. B. \>  zum Beispiel \\
\> ZWSF \>  Zirkulierende Wirbelschichtfeuerung \\
\end{tabbing}

% Symbolverzeichnis
%---------------------------------------------------------------------------------

\chapter*{Symbolverzeichnis}\thispagestyle{fancy}\markboth{Symbolverzeichnis}{Symbolverzeichnis}
\addcontentsline{toc}{chapter}{Symbolverzeichnis}

\section*{Formelzeichen}

\begin{tabbing}
\ \= \textbf{Symbol} \hspace{0.5cm} \= \textbf{Einheit} \hspace{1.5cm} \= \kill
\> $A$ \> $\mathrm{m^{2}}$ \> Fläche \\
\> $a$ \> $\mathrm{MPa}$ \> Kohäsionsdruck \\
\end{tabbing}

\section*{Griechische Symbole}

\begin{tabbing}
\ \= \textbf{Symbol} \hspace{0.5cm} \= \textbf{Einheit} \hspace{1.5cm} \= \kill
\> $\alpha$ \> $\mathrm{W\,/\,(m^{2} \cdot K)}$ \> Wärmeübergangskoeffizient\\
\> $\gamma$ \> - \> Aktivitätskoeffizient für die Flüssigphase \\
\end{tabbing}
% Abbildungsverzeichnis
%---------------------------------------------------------------------------------
\setcounter{lofdepth}{-1}
\listoffigures
\setcounter{lofdepth}{-1}
\chapter*{Einleitung}\thispagestyle{fancy}\markboth{Einleitung}{}
\addcontentsline{toc}{chapter}{Einleitung}

Die Firma Belectric GmbH wurde im 2001 gegr�ndet und hat sein Standort im Kolitzheim. Seit dem hat �ber 1,5 GWp Solarleistung weltweit installiert. Sie wurde somit eine Weltmarktf�hrer in den Bereich Installation von Freifl�chensolarkraftwerken. Es werden neue und innovative Technologien bei der Installation umgesetzt. Weltweit sind �ber 1600 besch�ftigte Menschen, die in Bereichen von Wartung und Anlagebau bis zu Forschen und Entwickeln beim Belectric arbeiten. 
Die Firma Adensis GmbH, mit dem Standort im Dresden, geh�rt zu den Entwicklungs- und Forschungsgruppen der Belectric GmbH. Sie wurde im 2006 gegr�ndet und seit dem betreibt ein Forschungszentrum f�r den Gebiet Photovoltaik. �ber 70 Mitarbeitern sind in den Bereichen Elektrotechnik, Maschinenbau, Physik und Chemie angestellt. Einen gr��eren Teil der Mitarbeiter bilden Studenten und ehemalige Studenten.  Zu den Aufgabenfeldern der Adensis GmbH geh�rt Durchf�hren von Testen und Analysen, Entwicklung neuer Technologien und Produkte sowie Kraftwerksbau. Meine 20-Wochige Praktikum wurde in den Abteilung Kraftwerkstechnik der Firma Adensis absolviert. 
Das Pflichtpraktikum war auf zwei Hauptthemen geteilt. Im Oktober habe ich mich mit einen Batterie-Management-System, der im Adensis entwickelt war,  besch�ftigt. Meine Aufgabe war einen Algorithmus f�r die B&R Steuerung entwickeln sollen. 
Ab November war ich mit Visualisierung von gesendeten Daten aus einer Kraftwerksanlage besch�ftigt. Das soll mittels einer Website, mit jeweiligen Grafischen Mitteln realisiert werden.

\include{EidesstattlicheErklarung}

\end{document}