\chapter*{Web-Visualisierung}\thispagestyle{fancy}\markboth{Web-Visualisierung}{}
\addcontentsline{toc}{chapter}{Web-Visualisierung}


\section{Besprechung mit Kollegen �ber Inhalt/Design }

\large{Es folgen Besprechungen, Brainstorming �ber das was wir eigentlich wollen, was m�chte wir damit errreichen, was unsere Ziel}

\section{Konzept f�r die Oberfl�che mittels Skizzen und grafischer Software)}

\large{Mittels grafischer Soft und Papier habe ich zu erst Konzepte f�r die Website erstellt, Ideen wie es aussehen soll, dann wieder besprechen}

\section{Nutzen von Bootstrap Frameworks, HTML, CSS f�r die Visualisierungen}

\large{Mit Website-tools, Sprachen, Frameworks habe ich erste Konzepte realisiert und die Elementen dargestellt}

\section{Mittels FTP die Seite online stellen}

\large{Kollegen haben einen Server erstellt, wo ich die Website veroffentlichen soll, damit die online verfugbar ist}

\section{Mit SQL Database arbeiten und Meta-Datenbanken erstellen}

\large{Verbinden von Daten aus SQL, Meta-Datenbank erstellt, welche die Allgemeine Infos und Daten zu Kraftwerk beinhaltet, Einen System erstellt und die Datenbank �ber SQL Sprache erstellt}

\section{Mit JavaScript, jQuery und PHP dynamische Umsetzung und mit SQL Datenbanken verbinden}

\large{Eine Dynamische Umsetzung erfolgt mittels PHP und andere.. es ist wichtig, dass die Seite dynamisch funktioniert, dass es automatisiert ist und wenn wir die SQL erweitern dann wird es auto. im Website angezeigt}

\section{Mit Hilfe von Diagramm-Bibliotheken die Daten aus Datenbanken visualisieren}

\large{Die Daten die im SQL liegen werden mittels Diagrammen visualisiert, es gibt mehrere Diagrammen-Bibliotheken, die man benutzen kann,  es war wichtig, dass die open-source, mit SQL arbeiten kann - dynamisch, viele Optionen wie Export, mehrere Linien anzeigen, mehrere Typen von Charts, Datum-anzeigen}
