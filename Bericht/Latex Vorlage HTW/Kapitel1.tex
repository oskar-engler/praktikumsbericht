
\chapter{Grundlagen}\thispagestyle{fancy}

\section{Allgemeines}
Allgemein orientieren sich Aufbau und Gestaltung der Diplomarbeit an den Normen zur Gestaltung wissenschaftlicher Arbeiten, siehe DIN 1421, 1422, 1502, 5008.
Die Diplomarbeit ist auf weißem Papier im Format A4 entsprechend den Richtlinien zur Textverarbeitung (DIN 5008, Duden) zu verfassen und zu binden.
Beleg- und Diplomarbeiten sind wissenschaftlich technische Dokumentationen, die allgemeinen Anforderungen bezüglich Struktur und Form entsprechen müssen. Sie sollen sich durch Klarheit im Ausdruck, guten Stil und einwandfreie Orthografie auszeichnen. Formulierungen sind sorgfältig zu wählen. Das betrifft auch die Verwendung fremdsprachiger Begriffe.
Die Arbeit ist logisch zu gliedern.
Der Schreibende soll einen anspruchsvollen wissenschaftlich-technischen Inhalt verständlich vermitteln. Dazu ist es erforderlich, sich immer wieder in die Position des Lesers zu versetzen und die Darstellung in dieser Hinsicht zu hinterfragen.
Das Bild, d. h. Prinzipskizze, Diagramm, Foto, Flussdiagramm, Tabelle etc., ist die „Sprache des Ingenieurs“ und sollte langen Erklärungen vorgezogen werden.
Ergebnisse müssen rückverfolgbar sein. Dazu sind die angewandten Methoden, Annahmen, Randbedingungen, experimentellen Einrichtungen und Programme sowie relevante Zwischenergebnisse zu nennen und in einem angemessenen Umfang zu erläutern.
Auch Berechnungen sind so zu dokumentieren, dass der Gutachter ihre Richtigkeit  überprüfen kann.
Die Darstellung sollte sich auf das Wesentliche konzentrieren und frei von allgemein bekannten Abhandlungen und „Füllstoff“ sein, die nur vom Thema ablenken und den „roten Faden“ verlieren lassen.
Der Arbeit ist eine schriftliche Erklärung beizufügen, dass sie selbständig angefertigt wurde und keine anderen als die angegebenen Unterlagen verwendet wurden.\\
Der Textteil der Arbeit besteht aus:
\begin{itemize}
\item Einleitung
\item Zielstellung
\item Lösungsweg
\item Hauptteil mit Unterabschnitten
\end{itemize}

\section{Einleitung}
Die Einleitung beschreibt den Aufbau der Arbeit, die Motive zu Ihrer Erstellung, die wissenschaftliche Herangehensweise an die Problemstellung sowie formale technische und ggf. rechtliche Rahmenbedingungen.

\section{Bindung}
Die Arbeit ist an der Professur in zweifacher Ausfertigung in gebundener Form (Ringbindung, feste Bindung), ausgedruckt auf weißem Papier (80 $g/m^{2}$), Format DIN A4, abzugeben. Die eigentlichen Seiten sind bei Ringbindung mit zwei leeren Deckblättern ($>$ 120 $g/m^{2}$) zu umschließen.\\
Für Diplom-, Master- und Bachelorarbeiten ist eine feste Bindung vorgeschrieben. Bitte die Exemplare beschriften (Rücken [Beschriftung 2 cm ab unterem Rand beginnen] und Deckel) mit DA Nr., Name, Vorname. \\
Muster hierzu können im Zweifelsfall in der Professur eingesehen werden.

\section{Elektronische Form}
Der ausgedruckten und gebundenen Arbeit ist eine CD/ DVD in einer Tasche auf der letzten Seite der Arbeit beizulegen. \\
Darauf ist die Arbeit als Datei, z. B. Word-Datei, alle erstellten elektronischen Modelle, Simulationen, Programme, Zeichnungen, Fotos, Vorlagen, Rechnungen usw. jeweils in weiter bearbeitbaren Formaten (SolidWorks, AutoCAD, Corel, Cosmos, Fluent, ComSol, jpg, bmp, MathCAD, ...) und recherchiertes Material, das Ihnen in elektronischer Form vorliegt (Zeitschriftenartikel, Scans, Auszüge aus Büchern, e-books, ...), mit abzulegen. \\
Sämtliche Dateien sind unverschlüsselt abzuspeichern.

\section{Formatierung}
\subsection{Allgemeines}
Alle Formatvorlagen und Seitenformate sind dieser Vorlage zu entnehmen. Fertigen Sie sich eine Kopie dieser Datei an. Benennen Sie diese dann um und löschen vorsichtig, Absatz für Absatz die Texte. Behalten Sie die sich automatisch aktualisierenden Verzeichnisse und Nummerierungen bei. Das vereinfacht den Einstieg in die Arbeit.

\subsection{Abkürzungen und physikalische Größen}
Sämtliche Abkürzungen sind mit ihrer ersten Verwendung einzuführen und entsprechend im Abkürzungsverzeichnis aufzunehmen. \\
Die Auflistung im Abkürzungsverzeichnis gilt auch für ``z. B'' oder ``etc.''. Abkürzungen, die für mehrere einzelne Wörter stehen. Mehrere einzelne Wörter wie ``z. B.'' oder ``u. a.'' sind auch in der Abkürzung durch ein Leerzeichen (empfohlen wird zur Vermeidung des ``Auseinanderziehens'' die Verwendung von $<$Backslash$>$ $<$Komma$>$ als Befehl für eine halbe Leerzeile) getrennt.\\
Genauso ist bei der Darstellung von Betrag und Einheit zu verfahren. (Beispiele: $5,67\, W/(m^{2}K^{4})$, $9,6 MJ/kg$, $12,0 \cdot 10^{-6}\,1/K$, $100\,^{\circ}\mathrm{C}$, $26\,\%$, jedoch: $90\,^{\circ}$).\\
Physikalische Größen sind in Maßeinheiten des internationalen Einheitensystems (SI) anzugeben, d. h. bei Verwendung historischer oder anglo-amerikanischer Quellen mit Nicht-SI-Einheiten (z B. atü, $mmWS$, $Torr$, $psi$, $^{\circ}F$, $kcal$, $PS$, $ft$, ...) entsprechend umzurechnen. Die Umrechnung ist in jedem Fall in der Arbeit zu dokumentieren.

\subsection{Absätze und Überschriften}
Unter Hauptüberschriften der Ebene 1, die mit Überschriften der Ebene 2 weiter unterteilt werden, wird kein Text eingefügt. Es sind entsprechende ``Einleitungs-Unterkapitel'' einzufügen.
Die maximale Gliederungstiefe für Diplomarbeiten beträgt 3 Ebenen. Entsprechend enthält das Inhaltsverzeichnis nur die Überschriften der Ebene 1 – 3.\\
Es ist der Blocksatz zu verwenden.

\subsection{Schriftgröße und Zeilenabstand}
In den jeweiligen Formatvorlagen sind die Schriftgröße (mind. 11) und der Zeilenabstand (1,5) geregelt.

\section{Rechtschreibung und Grammatik}
Die studentische Arbeit ist in der Regel in deutscher Sprache und dabei nach den Regeln der neuen deutschen Rechtschreibung anzufertigen.

\section{Quellen und Zitate}
Es ist ein Quellenverzeichnis zu führen. Wörtlich zitiert wird in Anfüh- rungszeichen mit einem Quellenverweis. \\
Beispiele: \\
Nach Meinung vom VERFASSER wird ``… wie hier zu sehen, wörtlich zitiert''.\\
Enthält das Zitat bereits einen abschließenden Satz-Schluss-Punkt, wird nach der schließenden Klammer der Quellenangabe kein Punkt gesetzt. \\
Die drei Punkte (…) werden in Zitaten verwendet, um alle Auslassungen zu kennzeichnen. 
Auch hier liefert der VERFASSER mit der Forderung ``… den bedingungslosem Einsatz … der Punkte … bei Auslassungen'', ein gutes Beispiel. \\
Bei drei Punkten am Ende eines Zitates wird ``… dringend der Punkt nach der Quellenangabe gesetzt …''.\\
Die Abkürzungen ``f.'' und ``ff.'' stehen für die Folgeseite bzw. die Folgeseiten.
Verweise zu Quellen sind in eckigen Klammern auszuführen, wie beispielsweise nachstehend für die 37. BImSchV. \cite{BImSchV}

\section{Fussnoten}
Kurzworte (z. B. REA\footnote{REA: gebräuchliches Kurzwort für Rauchgasentschwefelungsanlage} ), fremdsprachige (z. B. Betula pendula\footnote{Betula pendula: lat. für Hänge-Birke} ) und fachgebietsfremde Fachbegriffe sowie Markennamen (z. B. PERSIL\footnote{PERSIL: Vollwaschmittel, eingetragene Marke der Henkel AG, Düsseldorf} ) sind auf der Seite der ersten Nennung mittels einer Fußnote zu erklären. 

\section{Abbildungen und Tabellen}
Platzieren Sie alle Abbildungen und deren Beschriftung in Tabellen ``ohne Rahmen'' (2 Zeilen und 1 Spalte, bei zwei Stück nebeneinander liegenden Bildern: 2 Zeilen und 2 Spalten). \\
Beschriften Sie alle Abbildungen mit Bildunterschrift und Tabellen mit Tabellenüberschrift. \\
\\

\begin{figure} [h]
\centering
\includegraphics[scale=0.6]{images/Abbildung1.png}
\caption{Logo Professur}
\end{figure}

Am Ende der Arbeit wird je ein entsprechendes Verzeichnis eingefügt. Abbildungen aus dem Anhang werden hier nicht aufgeführt. Zentrieren Sie Ihre Grafiken und Tabellen.
\\
\begin{table} [h]
\begin{center}
\caption{Tabellenübersicht}
\begin{tabular}{|c|c|c|}
\hline 
Spalte 1 & Spalte 2 & Spalte 3 \\
\hline
Text & Text & Text \\
\hline
 & & \\
\hline
\end{tabular}
\end{center}
\end{table}
\\

Quelltexte und große Grafiken (im Format einer Seite) sollten in den Anhang verschoben werden, es sei denn der Lesefluss wird dadurch erheblich gestört. Dies ist z. B. der Fall, wenn die Grafik einen Abschnitt zusammenfasst. 

\section{Formeln}
Verwendung Sie zur Darstellung der Formeln den Mathematikmodus von LaTex. Die Nummerierung der Formeln sollte vorhanden sein.\\

\begin{equation} 
\dot{Q}_{Strahl} = C_{1,2} \cdot A_{1} \cdot \left[\left(\frac{T_{1}}{100}\right)^{4} - \left(\frac{T_{2}}{100}\right)^{4}\right]
\end{equation} 

\section{Veröffentlichung}
Studentische Arbeiten sind als wissenschaftliche Arbeiten öffentlich zu- gänglich. Sollte die Arbeit vertrauliche Informationen enthalten, die nicht veröffentlicht werden können, so ist der Textteil zumindest so zu verfassen, dass wissenschaftlicher Hintergrund, Lösungsweg und grundsätzliche Aussagen vom Leser nachvollziehbar sind. Dies ist z. B. durch normierte Diagramme und Prinzipdarstellungen möglich. Vertrauliche Daten sind in diesem Fall in einem nicht öffentlichen Anlageteil beizufügen, der nur den Gutachtern zur Prüfung der Arbeit zugänglich ist. (aus \cite{Dokument})

\section{Kommentare am rechten Blattrand}
Alle Kommentare dieser Vorlage sind in der Endversion Ihrer Arbeit zu löschen.


